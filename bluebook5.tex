\documentclass{blue-book}
\NoteNumber{5}

\usepackage{graphicx}
\usepackage{amssymb,amsmath,amsthm}
\usepackage{listings}

\let\sometime=\lozenge
\let\always=\Box

\title{A Dialogue System for identifying Disagreements in Deductive Reasoning}
\author{Louise A. Dennis}

\newtheorem{theorem}{Theorem}
\newtheorem{lemma}{Lemma}
\newtheorem{definition}{Definition}

\newcommand{\drule}[3]{\ensuremath{#1:#2 \rightarrow #3}}
\newcommand{\pt}[2]{\ensuremath{\langle #1,#2 \rangle}}
\newcommand{\intree}[2]{\ensuremath{#1 \in_t #2}}
\newcommand{\node}[2]{\ensuremath{(#1, #2)}}
\newcommand{\edge}[2]{\ensuremath{#1 \mapsto #2}}
\newcommand{\deduction}[3]{\ensuremath{{\cal D}(#1, #2, #3)}}

\usepackage[colorinlistoftodos]{todonotes}
% \presetkeys%
%     {todonotes}%
%     {inline}{}
\newcommand{\ldnote}[1]{\todo[linecolor=yellow,backgroundcolor=yellow!25,bordercolor=yellow]{LD: #1}}
\newcommand{\louiseinline}[1]{\todo[inline, color=yellow!50]{Louise: #1}}
\newcommand{\yxnote}[1]{\todo[linecolor=yellow,backgroundcolor=green!25,bordercolor=green]{YX: #1}}
\newcommand{\yifaninline}[1]{\todo[inline, color=green!50]{Yifan #1}}

\begin{document}
\maketitle

Yifan is working on a dialogue based system for explaining reasoning in rules-based systems.  We have worked through various versions of this with a Prolog implementation, starting out treating proofs as acyclic graphs with (we hoped) some kind of completeness property (i.e., the graph would capture all necessary information about the proof) and moving slowing to an idea around proof trees with Prolog-style negation as failure.

Our starting point is two ``players'' (assumed to be some deductive system and a user).  Each possesses some set of facts ($F$) and a set of rules ($R$) of the form \emph{if ...  then ...} and use these to deduce whether some conclusion, $c$, is true or false.  Deductions are represented as trees.  Each node in the tree is labelled with the fact or conclusion deduced (represented as a first-order formula) and either the label \texttt{initial} indicating an initial fact, \texttt{unprovable} meaning the fact or conclusion can not be proved (and so is deduced to be false), or the rule that was used to deduce term labelling the node. 

When the players disagree about some conclusion they engage in a dialogue.  Each player can ask \emph{why} a particular node is believed in which case they are informed that it was either an initial fact, or it was deduced from its parent nodes using the rule.  A player can also ask \emph{why not} questions -- i.e., why does some conclusion is unprovable.  In this case the system turns this around and asks them why they believe that it is.

The intention is that through this dialogue the user can understand either that the system did (or did not) believe some initial fact that the user does not (or does) believe or that the system and user's rules differ.

Note we assume that both players reason correctly and do not consider the case where mistakes are made in the reasoning process.

\begin{sloppypar}
This note aims at trying to establish a formal definition of the dialogue process and formal properties of that definition (particularly that it terminates with a disagreement found).  Our starting point was~\cite{DennisAAMAS21} which does something similar in the context of BDI-style reasoning.
\end{sloppypar}

\section{Preliminaries}
We have some language of terms, ${\cal L}$, defined in a standard way.

\begin{itemize}
\item We have a set of labels $L$ which include two special labels, $initial$ and $unprovable$. 

\item We have sets of initial facts, $F$, these consist of positive literals in ${\cal L}$.

\item We have sets of rules, $R$.  A rule is a horn clause consisting of a non-empty set of literals in ${\cal L}$ (the antecedants, $A$), and a consequent, a positive literal $C \in {\cal L}$ , and a label $l \in L \backslash \{initial, unprovable\}$.  We write a rule as $\drule{l}{A}{C}$.  We assume that labels in $R$ are unique -- i.e., there is only one rule labelled $l$ in any set of rules, $R$.  Note that we don't need to label rules for our system to work, but labels are a useful convenience when referring to rules.
\end{itemize}

\begin{definition}{\bf (Positive and Negative Literals in a set)}
For a set of literals, $A$, $$pos(A) = \{l: l \text{ is a positive literal } \wedge l \in A\}$$ (the set of positive literals appearing in $A$).

For a set of literals, $A$, $$neg(A) = \{l: l \text{ is a positive literal } \wedge \neg l \in A\}$$ (the set of literals appearing negatively in $A$).
\end{definition}

\subsection{Proof Trees}

\begin{definition}{\bf (Proof Tree)}
\label{def:proof_tree}
A proof tree is a directed rooted tree written $\pt{N}{E}$, where $N$ is the set of nodes in the tree, and $E \subseteq N \times N$ is the set of edges.  Each node $n \in N$ consists of a pair of a ground positive literal, $t \in {\cal L}$ and a label, $l \in L$ written $\node{t}{l}$.  An edge between two nodes $n_1$ and $n_2$ is written as $\edge{n_1}{n_2}$.

We use standard terminology so the root of a proof tree is the single node, $n$ such that there is no edge $\edge{n'}{n}$.  A node, $n$'s, parent nodes are the set of nodes, $n'$, such that there exists an edge $\edge{n}{n'}$.  A node, $n$'s, parent trees are the set of sub-trees with a parent of $n$ as their root.

If $\node{t}{l}$ is the root node of a tree, then we refer to $t$ as the \emph{root term} of the tree.
\end{definition}

\begin{definition}{\bf Provable and Unprovable in $T$}
If $\langle N, E \rangle = T$ is a proof tree and $t$ is a ground positive literal in ${\cal L}$.  We say:
\begin{itemize}
\item $t$ is \emph{provable} in $T$ iff there exists a node $\node{t}{l} \in N$ such that $l \neq unprovable$.
\item $t$ is \emph{unprovable} in $T$ iff $\node{t}{unprovable} \in N$.
\item $t$ is \emph{undecided} in $T$ iff there is no node $\node{t}{l} \in N$.
\end{itemize}
%% Note that because the definition of a proof tree states that for any ground positive literal, $t$, there is at most one node $\node{t}{l} \in N$ it is not possible for a term to be both provable and unprovable in any proof tree.
\end{definition}

\begin{definition}{\bf (Proof Tree for $F$ and $R$)}
A \emph{proof tree}, $T$, for a set of facts, $F$, and rules, $R$ is defined recursively as follows:

\begin{itemize}
\item $\pt{\{\node{t}{initial}\}}{\emptyset}$ is proof tree for $F$ and $R$ iff $t \in F$
\item $\pt{\{\node{t}{unprovable} \}}{\emptyset}$, is a proof tree for $F$ and $R$ iff no proof tree, $T'$, for $F$ and $R$ exists such that $t$ is provable in $T'$
\item If $E \neq \emptyset$ then a proof tree $T = \pt{N}{E}$ with root node $\node{t}{l}$ is a proof tree for $F$ and $R$ iff:

\begin{itemize}
\item The parent trees of $\node{t}{l}$ are all proof trees for $F$ and $R$
\item There exists a rule, $\drule{l}{A}{C} \in R$ and a substitution, $\theta$ for the free variables in $A$ and $C$ such that $C\theta = t$ and $t \not\in F$.
\item If $\node{t'}{l'}$ is a parent of $\node{t}{l}$ in $T$ then either 
\begin{itemize} 
\item $\exists t_i \in pos(A). \: t_i\theta = t'$ and $l' \neq unprovable$ or, 
\item $\exists t_i \in neg(A).  \: t_i\theta = t'$ and $l' = unprovable$.
\end{itemize}
\item For all $t'$ such that $t' \in A\theta$ there exists a unique label, $l'$ such that $\node{t'}{l'}$ is a parent node of $\node{t}{l}$ in $T$.  Note that because $A \neq \emptyset$ this means that if any proof tree, $T$, for $F$ and $R$ that contains a node $\node{t}{l}$ where $l \not\in \{initial,unprovable\}$ contains more than one node and at least one edge.
\end{itemize}
\end{itemize}
\end{definition}

We note (though we don't prove) that a proof tree with some statement $t$ at its root (either as a provable or unprovable statement) can be constructed from $F$ and $R$ by standard backwards reasoning in a Prolog style.

I'm going to stop talking about substitutions, $\theta$ etc., from this point.  They make everything harder to read and I don't think they make much difference to correctness.  

We will also assume that we are only dealing with sets of facts and rules that always terminate when attempting to find a proof tree with some term $t$ as the root.

\begin{theorem}
\label{thm:unprovable}
If there exists a proof tree, $T$, for $F$ and $R$ and a term $t$ which is unprovable in $T$ then there is no proof tree for $F$ and $R$ in which $t$ is provable.
\end{theorem}
\begin{proof}
This is a very simple proof by induction.  In the base case, if $t$ is unprovable in $T$ then $T = \pt{\{\node{t}{unprovable} \}}{\emptyset}$ and there is no proof tree for $F$ and $R$ in which $t$ is provable by definition.  In the step case $t$ is unprovable in some $T$ with root node $\node{t}{l}$ where $\drule{l}{A}{C} \in R$.  Since no rule is labelled $unprovable$ by definition, $t$ must be unprovable in one of the sub-trees of $T$ and so, by the induction hypothesis, there is no proof tree for $F$ and $R$ in which $t$ is provable.
\end{proof}
\begin{theorem}
\label{thm:rules}
If there exists a proof tree, $\pt{N}{E}$, for $F$ and $R$ and a node $\node{t}{l} \in N$ where $l \not\in \{initial,unprovable\}$ then there exists a rule $\drule{l}{A}{C} \in R$.
\end{theorem}
\begin{proof}
Again this is a very simple proof by induction.  In the base case single node trees are labelled with either $initial$ or $unprovable$ so the theorem is trivially true.  Otherwise in the step case, if the root of the tree is the $\node{t}{l}$ then there exists a rule, $\drule{l}{A}{C} \in R$ by definition otherwise $\node{t}{l}$ is in one of the sub-trees and the existence of a rule $\drule{l}{A}{C} \in R$  follows from the induction hypothesis.
\end{proof}
\section{Problem Statement}

We need to formalise the idea of a disagreement between the user and the reasoning system.  We will simplify the problem to identifying a \emph{cause} of a difference between two proof trees (this is obviously a massive simplification of why a user might want more information about a reasoning system's deduction, but it is a solid first step).

\begin{definition}
We formalise a deduction as a tuple $\deduction{F}{R}{{\cal  T}}$ where $F$ is a set of initial facts, $R$ a set of rules and ${\cal T}$ is a set of proof trees for $F$ and $R$.  We will refer to $F$ as the \emph{deduction facts}, $R$ as the \emph{deduction rules} and ${\cal T}$ as the \emph{deduction trees}.
\end{definition}

Our problem is, given two initial deductions $\deduction{F_1}{R_1}{\{T_1\}} \neq \deduction{F_2}{R_2}{\{T_2\}}$ such that $T_1$ and $T_2$ have the same root term but that root term is provable in one and unprovable in the other, can we identify at least one fact, $t$ such that $t \in F_1$ and $t \not\in F_2$ (or vice versa) or at least one rule $r$ such that $r \in R_1$ and $r\not\in R_2$.  Note that we can trivially identify the differences if we have full access to $F_1$, $F_2$, $R_1$, $R_2$ etc., So our problem is whether we can find out by an incremental dialogue process where the starting information is simply a root term that is provable in one deduction tree and not in the other.  We assume that rules with the same label in $R_1$ and $R_2$ are identical -- i.e., if $\drule{l}{A_1}{C_1} \in R_1$ and $\drule{l}{A_2}{C_2} \in R_2$ then $A_1 = A_2$ and $C_1 = C_2$.  This means we can use rule labels as proxies for the rules themselves rather than having to match antecedants and consequents.

\section{A Dialogue}

A dialogue is a sequence of moves taken by two players.  $P_1$ knows all the information in $D^1_0 = \deduction{F_1}{R_1}{\{T_1\}}$ while $P_2$ knows all the information in $D^2_0 = \deduction{F_2}{R_2}{\{T_2\}}$.  The two players gradually build up a model of how the other player has reasoned.  This model consists of four sets $Y_{ij}$, $F_{ij}$, $N_{ij}$ and $YR_{ij}$: 
\begin{itemize}
\item $Y_{ij}$ consists of terms $t$ that $P_i$ has established that $P_j$ believes to be true.  We refer to $Y_{ij}$ as the \emph{opponent belief set}.
\item $F_{ij}$ consists of terms $t$ that $P_i$ has established that $P_j$ had as an initial fact.  Note that $F_{ij} 
\subseteq Y_{ij}$.  We refer to $F_{ij}$ as the \emph{opponent fact set}.
\item$N_{ij}$ consists of terms $t$ that $P_i$ has established that $P_j$ does not believe to be true.  We refer to $N_{ij}$ as the \emph{opponent disbelief set}.
\item $YR_{ij}$ consists of rule labels $l$ that $P_i$ has established label on of $P_j$'s rules.  We refer to $YR_{ij}$ as sthe \emph{opponent rule set}.
\end{itemize}
At the same time they may need to add additional proof trees to their deduction trees in response to questions asked by the other player.

There are six possible statements that can be made in the course of a dialogue:
\begin{enumerate}
\item $\mathit{different\_fact(t, i, j)}$ -- $i$ has $t$ as an initial fact and $j$ does not.
\item $\mathit{different\_rule(\drule{l}{A}{C}, i, j)}$ -- $i$ has $\drule{l}{A}{C}$ as a rule and $j$ does not.
\item $initial(t)$ -- $t$ is an initial fact for the Player.
\item $\drule{l}{a}{t}$ -- the player deduced $t$ from the terms in $a$ using the rule labelled $l$
\item $why(t)$ - why do you believe $t$?
\item $whynot(t)$ -- why don't you believe $t$
\end{enumerate}
The first two statements terminate the dialogue.


The starting point for the dialogue is the root term $t$ of $T_1$ and $T_2$ which, for the sake of argument, we'll say is provable in $T_2$ and unprovable in $T_1$.  So $t$ is something one of them has deduced and the other hasn't.  We don't worry at this point about how the players have established this initial disagreement.  

\begin{sloppypar}
\begin{definition}{\bf (Player State)}
The \emph{state} of player $i$ at statement $k$ in a dialogue with player $j$ is $S^i_k = \langle  D_i, Y_{ij}, F_{ij}, N_{ij}, YR_{ij} \rangle$ where $D_i$ is a deduction (referred to as $i$'s deduction), and $Y_{ij}, F_{ij}, N_{ij}, YR_{ij}$ are sets of terms representing $i$'s opponent belief set, fact, set, disbelief set and rule set respectively.
\end{definition}

\begin{definition}{\bf (Initial Player State)}
The initial state of  Player $i$ in a dialogue is ether $\langle \deduction{F_i}{R_i}{\{\pt{\node{t}{unprovable}}{\emptyset}\}}, \{t\}, \emptyset, \emptyset, \emptyset \rangle$ or  $\langle \deduction{F_i}{R_i}{\{T_i\}}, \emptyset, \emptyset, \{t\}, \emptyset \rangle$ where $t$ is the root term of $T_i$.
\end{definition}
\end{sloppypar}



\begin{definition}{\bf (Dialogue State)}
$S_{k}$ is the state of the dialogue after the utterance of the $k$th statement. It consists of the two player states, the last dialogue statement, $stmt$, and whose turn it is $i$.  $S_{k} = \langle S^1_{k}, S^2_{k}, stmt, i \rangle$
\end{definition}

\begin{definition}{\bf (Dialogue)}
A dialogue is a sequence of dialogue states $S_0, \ldots, S_n$.

$S_0 = \langle S^1_0, S^2_0, stmt_0, i \rangle$ where $S^1_0 = \langle D^1_0, \{t\}, \emptyset, \emptyset, \emptyset \rangle$ is an initial player state, $S^2_0 = \langle D^2_0, \emptyset, \emptyset, \{t\}, \emptyset \rangle$ is an initial player state, and either $i = 1$ and $stmt_0 = whynot(t)$ (Player 2 started the dialogue by asking Player 1 why they don't believe $t$ and it is now Player 1's turn) or $i = 2$ and $stmt_0 = why(t)$ (Player 1 started the dialogue by asking Player 2 why they believe $t$ and it is now Player 2's turn). 

If $S_k = \langle S^1_{k}, S^2_{k}, stmt_k, i_k\rangle$ is the state of a dialogue at utterance $k$ then $S_{k+1} = \langle S^1_{k + 1}, S^2_{k + 1}, stmt_{k + 1}, i_{k + 1}\rangle$ is a legal next state in the dialogue if the individual parts of the state are as follows:

\begin{itemize}

\item  {\bf ($S^i_{k+1}$ where $i = i_k$)} Recall that $S^i_k =  \langle D_i, Y_{ij}, F_{ij}, N_{ij}, YR_{ij} \rangle$
\begin{enumerate}
\item If $stmt_k = initial(t)$ then $S^i_{k+1}= \langle D_i, Y_{ij} \cup \{t\}, F_{ij} \cup \{t\}, N_{ij}, YR_{ij} \rangle$ 
($P_i$ adds $t$ to $Y_{ij}$ and $F_{ij}$).  
\item If $stmt_k = \drule{l}{a}{t}$ and $l  \in R_i$ then $S^i_{k + 1} = \langle D_i, Y_{ij} \cup pos(a), F_{ij}, N_{ij} \cup neg(a), YR_{ij} \cup \{l\} \rangle$
($P_i$ adds all the positive literals in $a$ to $Y_{ij}$ (these are things the other player believes) and all the negative literals in $a$ to $N_{ij}$ (these are all the things the other player does not believe) and adds $l$ to $YR_{ij}$ -- this is a rule the other player has). 
\item If $stmt_k = why(t)$ and $t$ is provable in one of the deduction trees in $D_i$ then $S^i_{k + 1} = \langle D_i, Y_{ij}  , F_{ij}, N_{ij} \cup \{t\}, YR_{ij} \rangle$ ($P_i$ adds $t$ to $N_{ij}$ (the other player doesn't believe $t$)). 
\item If $stmt_k = why(t)$, $D_i = \deduction{F_i}{R_i}{{\cal T}}$ and $t$ is undecided in all the proof trees in ${\cal T}$, and there exists a proof tree for $F_i$ and $R_i$ in which $t$ is provable then $P_i$ generates a new proof tree for $F_i$ and $R_i$, $T$, with $t$ as its root term and  $S^i_{k + 1} = \langle \deduction{F_i}{R_i}{{\cal T} \cup T}, Y_{ij}  , F_{ij}, N_{ij} \cup \{t\}, YR_{ij} \rangle$ ($P_i$ adds $t$ to $N_{ij}$ (the other player doesn't believe $t$) at the same time $P_i$ generates a deduction as evidence for why it believes $t$). 
\item If $stmt_k = why(t)$, $D_i = \deduction{F_i}{R_i}{{\cal T}}$ and $\langle \{\node{t}{unprovable}\}, \emptyset \rangle$ is a proof tree for $F_i$ and $R_i$ then  $S^i_{k + 1} = \epsilon$ and the dialogue terminates in an error state (This state should not occur in a legal dialogue -- $P_i$ has been asked why it thinks something is true when in fact $P_i$ thinks it is false).
\item If $stmt_k = whynot(t)$, $D_i = \deduction{F_i}{R_i}{{\cal T}}$ and $t$ is unprovable in some proof tree in ${\cal T}$ then $S^i_{k + 1} = \langle D_i, Y_{ij} \cup \{t\}, F_{ij}, N_{ij} , YR_{ij} \rangle$  ($P_i$ adds $t$ to $Y_{ij}$ (the other player believes $t$)). 
\item If $stmt_k = whynot(t)$, $D_i = \deduction{F_i}{R_i}{{\cal T}}$, $t$ is undecided in all the proof trees in ${\cal T}$ or $t$ is provable in some proof tree in ${\cal T}$ then  $S^i_{k + 1} = \epsilon$ and the dialogue terminates in an error state (This state should not occur in a legal dialogue -- $P_i$ has been asked why it thinks something is false when in fact $P_i$ thinks it is true or has never tried to prove it).
\end{enumerate}

\item {\bf ($S^j_{k+1}$ where $j \neq i_k$)} $S^j_{k+1} = S^j_{k}$ (the state of the player who's turn it is not, is unchanged at the next state in the dialogue)

\item {\bf ($stmt_{k+1}$)} (Legal  moves for player $P_{i_k}$ in their turn in the dialogue). In what follows $D_{i_k} = \deduction{F_{i_k}}{R_{i_k}}{{\cal  T}_{i_k}}$.

\begin{enumerate}
\item $stmt_{k + 1} = initial(t)$ is legal iff $stmt_{k} = why(t)$ and $t \in F_{i_k}$
\item $stmt_{k + 1} = \drule{l}{a}{t}$ is legal iff:
\begin{itemize}
\item$stmt_{k} = why(t)$, 
\item $t \not\in F_{i_k}$,
\item there exists a proof tree $\langle N, E\rangle \in {\cal T}_{i_k}$ such that $\node{t}{l} \in N$
% \item For every literal  $t_i \in A$ there exists a node $\node{t_i}{l_i} \in N$ such that $\edge{\node{t}{l}}{\node{t_i}{l_i}} \in E$
% \item If $t_i \in pos(A)$ then it is provable in $\langle N, E \rangle$ 
% \item If $t_i \in neg(A)$ then it is unprovable in $\langle N, E \rangle$
\end{itemize}
\item $stmt_{k + 1} = whynot(t)$ is legal iff 
\begin{itemize}
\item $stmt_{k} \neq why(t)$
\item $\forall l. l \leq k \rightarrow stmt_{l} \neq whynot(t)$ (this question has not been asked before)
\item there exists a $T \in {\cal T}_{i_k}$ such that $t$ is provable in $T$
\item $t \in N_{{i_k}j}$  (Player ${i_k}$ identifies a term $t$ that it believes and it has established the other doesn't  and asks why not).
\end{itemize}
\item $stmt_{k + 1} = why(t)$ is legal if $stmt_{k} =  whynot(t)$.
\item $stmt_{k + 1} = why(t)$ is legal if 
\begin{itemize}
\item $stmt_{k} \neq whynot(t)$,
\item $\forall l. l \leq k \rightarrow stmt_{l} \neq why(t)$ (this question has not been asked before)
\item there exists a $T \in {\cal T}_{i_k}$ such that $t$ is unprovable in $T$ and,
\item $t \in Y_{{i_k}j}$  (Player ${i_k}$ identifies a term $t$ that it does not believe and it has established the other does  and asks why).
\end{itemize}
\item $stmt_{k + 1} = \mathit{different\_fact}(t, j, {i_k})$ is legal iff $t \in F_{{i_k}j}$ and $t \not\in F_{i_k}$ (Player ${i_k}$ has identified that $t$ was an initial fact for Player $j$ but not for Player ${i_k}$)
\item $stmt_{k + 1} = \mathit{different\_fact}(t, {i_k}, j)$ is legal iff  $t \in N_{{i_k}j}$ and $t \in F_{i_k}$  (Player ${i_k}$ has identified that $t$ was an initial fact for Player ${i_k}$ but not for Player $j$)
\item $stmt_{k+1} = \mathit{different\_rule}(l, j, {i_k})$ is legal iff $l \in YR_{{i_k}j}$ and there is no rule $\drule{l}{A}{C} \in R_{i_k}$ (Player ${i_k}$ has identified that $l$ is a rule for  Player $j$ but not for Player ${i_k}$)
\item no other statements are legal.
\end{enumerate}
\item {\bf ($i_{k+1}$)} $i_{k + 1} \in \{1, 2\}$ and $i_{k + 1} \neq i_k$.  (Next player's turn).
\end{itemize}
\end{definition}

Note that there are some situations where several responses are legal in which case the player may select one.  At a later date we may look at heuristics for selecting responses.

\section{Proofs}

\begin{lemma}
\label{lem:initial}
If the current state of a dialogue is $\langle S^1_{k}, S^2_{k}, initial(t), i \rangle$,  $i \neq j$ and $\deduction{F_j}{R_j}{{\cal T}_j}$ then  $t \in F_j$ where $F_j$ and $t$ is provable in some $T \in {\cal T}_j$.% and $t$ is provable in $T_j$ where $j$'s deduction is $\langle F_j, R_j, T_j \rangle$.
\end{lemma}
\begin{proof}
If it is player $i$'s turn then $initial(t)$ was uttered by player $j$.  This could only be uttered if $t \in F_j$.  

Morever suppose $\deduction{F_j}{R_j}{{\cal T}_j'}$ were $j$'s deduction in the previous state. If $t$ is provable in some $T \in {\cal T}_j'$ then ${\cal T}_j = {\cal T}_j'$ and so $t$ is provable in some $T \in {\cal T}_j$.  If $t$ were undecided in all proof trees in ${\cal T}_j'$ then because a proof tree for $F_j$ and $R_j$ exists with $t$ as its root (namely $T = \pt{\{\node{t}{initial}\}}{\emptyset}$) then ${\cal T}_j = {\cal T}_j' \cup \{T\}$ so  $t$ is provable in some $T \in {\cal T}_j$.% Because $t \in F_j \rightarrow \intree{t \in T_j$  then $t \in G_j$.
\end{proof}
\begin{lemma}
\label{lem:drule}
If the current state of a dialogue is $\langle S^1_{k}, S^2_{k}, \drule{l}{a}{t}, i \rangle$, $i \neq j$ and $\deduction{F_j}{R_j}{{\cal T}_j}$ is $j$'s current deduction then there exists at least one proof tree, $T_j \in {\cal T}_j$  such that $\node{t}{l}$ is a node in $T_j$ , $\drule{l}{a}{t} \in R_j$ and for all $t\in pos(a)$, $t$ is provable in $T_j$; for all $t \in neg(a)$  then $t$ is unprovable $T_j$; and $l \in R_j$.
\end{lemma}
\begin{proof}
If it is player $i$'s turn then $\drule{l}{a}{t}$ was uttered by player $j$.  This could only be uttered if $n = \langle t, l \rangle \in N$ for some proof tree $\pt{N}{E} \in {\cal T}_j$.  The rest follows from the definition of a proof tree for $F_j$ and $R_j$.
\end{proof}
\begin{lemma}
\label{lem:whynot}
If the current state of a dialogue is $\langle S^1_{stmt}, S^2_{stmt}, whynot(t), i \rangle$, $i \neq j$ and $\deduction{F_j}{R_j}{{\cal T}_j}$ is $j$'s current deduction then $t$ is provable in some $T \in {\cal T}_j$.
\end{lemma}
\begin{proof}
If it is player $i$'s turn then $whynot(t)$ was uttered by player $j$.  Either this was the opening statement of the dialogue in which case $t$ is the root node of the only proof tree in ${\cal T}_j$.  Otherwise this could only be uttered if there exists a $T \in {\cal T}_j$ such that $t$ is provable in $T$.
\end{proof}

The equivalent lemma to the above for $why(t)$ is a corollary of the following theoarem.

\paragraph{State updates are correct} We need to establish athat for instance $t \in F_{ij}$ iff $t \in F_j$ (i.e., player $i$ only decides player $j$ has $t$ as an initial fact if player $j$ does indeed have $j$ as an initial fact).  The same for $Y_{ij}$, $N_{ij}$ etc. etc.  This involves going through the statements $j$ can make that will lead to $i$ updating those states.  We also want to show that the error states $\epsilon$ never occur.

\begin{theorem}
Given two dialogue participants $i$ and $j$, where $\deduction{F_j}{R_j}{{\cal T}_j}$ is $j$'s deduction and $F_{ij}$ is $i$'s opponent fact set, $F_{ij} \subseteq F_j$.  That is $i$'s model of $j$'s initial facts is a subset of $j$'s actual initial facts.
\end{theorem}
\begin{proof}
We will argue by induction considering each time something is added to $F_{ij}$.

{\bf Base case}.  In the initial state of the dialogue $F_{ij} = \emptyset$ therefore $F_{ij} \subseteq F_j$.

{\bf Step case}.  A term $t$ is added to $F_{ij}$ when the previous utterance $stmt = initial(t)$.  From Lemma~\ref{lem:initial} we know that in this case $t \in F_j$.  Therefore if $F_{ij} \subseteq F_{j}$ before $t$ is added to it then $F_{ij} \cup \{t\} \subseteq F_{j}$.

\end{proof}
\begin{theorem}
\label{thm:Ycorrect}
Given two dialogue participants $i$ and $j$ where $\deduction{F_j}{R_j}{{\cal T}_j}$ is $j$'s deduction and $Y_{ij}$ is $i$'s opponent belief set, then all terms $t \in Y_{ij}$ are provable in some $T \in {\cal T}_j$.  That is $i$'s model of $j$'s beliefs is a subset of the provable terms in $j$'s deduction.
\end{theorem}
\begin{proof}
We will argue by induction considering each time something is added to $Y_{ij}$.

{\bf Base case}.  There are two possible initial states of $Y_{ij}$ given initial deductions $\deduction{F_i}{R_i}{\{T_i\}}$ and $\deduction{F_j}{R_j}{\{T_j\}}$ for the two players.  Either $Y_{ij} = \{t\}$ where $t$ is a term that is provable in $T_j$ but not in $T_i$.  In this case, since $t$ is provable $T_j$ we are done.  Otherwise $Y_{ij} = \emptyset$ and so there are no terms in $Y_{ij}$ and we are also done.

{\bf Step case}.  A term $t$ is added to $Y_{ij}$ following a statment $s$ when:
\begin{enumerate}
\item  $s = initial(t)$.  From Lemma~\ref{lem:initial} we know that  $t$ is provable in some $T \in {\cal T}_j$.  
\item $s = \drule{l}{a}{r}$ and $t$ is a positive literal in $a$.  From Lemma~\ref{lem:drule} we know that all terms in $pos(a)$ are provable in  some $T \in {\cal T}_j$.
\item $s = whynot(t)$.  From Lemma~\ref{lem:whynot} we  know that in this case $t$ is provable in some $T \in {\cal T}_j$.  
\end{enumerate}
\end{proof}
{\bf Corollary} \emph{If the current state of a dialogue is $\langle S^1_{stmt}, S^2_{stmt}, why(t), i\rangle$ and $\deduction{F_i}{R_i}{{\cal T}_i}$ is $i$'s deduction then $t$ is provable in some $T \in {\cal T}_i$.}  This follows because $j$ can only ask $why(t)$ if either a) it is the initial state of the dialogue in which case $t$ is the root term of the only deduction tree in ${\cal T}_i$ by definition of initial player state or b) $t \in Y_{ji}$ where $Y_{j}$ is $j$'s opponent belief set and we now know that all terms in $Y_{ji}$ are provable in some $T \in {\cal T}_i$  or c) in response to $whynot(t)$ which can only be asked if $t$ is provable in some $T \in {\cal T}_i$.

{\bf Corollary 2} \emph{If the current state of a dialogue is $\langle S^1_{stmt}, S^2_{stmt}, why(t), i\rangle$ then $S^i_{stmt} \neq \epsilon$ -- i.e., the error state never arises in response to a why question}. This follows directly from the previous corollary since $t$ is provable in some proof tree $F_i$ and $R_i$.

\begin{theorem}
\label{thm:Ncorrect}
Given two dialogue participants $i$ and $j$, where $\deduction{F_j}{R_j}{{\cal T}_j}$ is $j$'s deduction and $N_{ij}$ is $i$'s opponent disbelief set, then all terms $t \in N_{ij}$ are unprovable in some proof tree $T \in {\cal T}_j$.
%% for $F_j$ and $R_j$ in which any term in $N_{ij}$ is provable and there exists $T \in {\cal T}_j$ where $T$ is unprovable.  That is $i$'s model of things $j$ doesn't believe doesn't contain anything in any proof tree $j$ could construct.
\end{theorem}
\begin{proof}
We will argue by induction considering each time something is added to $N_{ij}$.

{\bf Base case}.  There are two possible initial states of $N_{ij}$ given initial deductions $\deduction{F_i}{R_i}{\{T_i\}}$ and $\deduction{F_j}{R_j}{\{T_j\}}$ for the two players.  Either $N_{ij} = \{t\}$ where $t$ is a term that is provable in $T_i$ and unprovable $T_j$ and we are done.  Otherwise $N_{ij} = \emptyset$.

{\bf Step case}.  A term $t$ is added to $N_{ij}$ following a statement $s$ by $j$ when:
\begin{enumerate}
\item $s = \drule{l}{a}{r}$ and $t \in neg(a)$.  From Lemma~\ref{lem:drule} we know that all terms in $neg(a)$ are unprovable in some $T \in {\cal T}_j$.  
\item $s = why(t)$.  There are three occasions where $j$ can ask $why(t)$.  If $j$ has initiated the dialogue then ${\cal T}_j = \{\pt{\node{t}{unprovable}}{\emptyset}$ so $t$ is unprovable in the only tree in ${\cal T}_j$.  Secondly it is uttered in response to $whynot(t)$ asked by $i$ in this case $t$ was already in $N_{ij}$ and so is not added to the set, so we can ignore this case.  The last case is when $t \in Y_{ji}$ where $Y_{ji}$ is $j$'s opponent belief set and there exists a $T \in {\cal T}_j$ such that $t$ is unprovable in $T$.
\end{enumerate}
\end{proof}

{\bf Corollary} \emph{If the current state of a dialogue is $\langle S^1_{stmt}, S^2_{stmt}, whynot(t), i\rangle$ and $\deduction{F_i}{R_i}{{\cal T}_i}$ is $i$'s deduction then there is no proof tree for $F_i$ and $R_i$ in which $t$ is provable.}  This follows because $j$ can only ask $whynot(t)$ if $t \in N_{ji}$ so there is some proof tree for $F_i$ and $R_i$ in which $t$ is unprovable.  From this it follows that there is no proof tree for $F_i$ and $R_i$ in which $t$ is provable for any $t \in N_{ji}$ by Theorem~\ref{thm:unprovable}.

{\bf Corollary 2} \emph{If the current state of a dialogue is $\langle S^1_{stmt}, S^2_{stmt}, whynot(t), i\rangle$ then $S^i_{stmt} \neq \epsilon$ -- i.e., the error state never arrises in response to a why not question.} This follows directly from the previous corollary $t$ is not provable in any proof tree $F_i$ and $R_i$.

{\bf Corollary 3} \emph{If the current state of a dialogue is $\langle S^1_{stmt}, S^2_{stmt}, why(t), i\rangle$ and $\deduction{F_j}{R_j}{{\cal T}_j}$ is $j$'s deduction then $t$ is not provable in any proof tree for $F_j$ and $R_j$.}  This follows because $j$ can only ask $why(t)$ if either $i$ asked $whynot(i)$ in which case this is true from our first corollary or if there is a $T \in {\cal T}_j$ in which $t$ in unprovable in which case  there is no proof tree for $F_j$ and $R_j$ in which $t$ is provable by Theorem~\ref{thm:unprovable}.

\begin{theorem}
Given two dialogue participants $i$ and $j$, where $\deduction{F_j}{R_j}{{\cal T}_j}$ is $j$'s deduction and $YR_{ij}$ is $i$'s opponent rule set, then $\forall l. l \in R_{ij} \implies \exists a, c. \: \drule{l}{a}{c} \in R_j$ \end{theorem}
\begin{proof}
We will argue by induction considering each time something is added to $YR_{ij}$.

{\bf Base case}.  In the initial state of the dialogue $YR_{ij} = \emptyset$ and the theorem is trivially true.


{\bf Step case}.  A rule label $l$ is added to $YR_{ij}$ when the previous statement is $\drule{l}{a}{r}$ .  From Lemma~\ref{lem:drule} we know that there is at least one proof tree, $T \in {\cal T}_j$ such that $\node{t}{l}$ is a node in $T_j$.  Since $T$ is a proof tree for $F_j$ and $R_j$ there exists a rule  $\drule{l}{a}{c} \in R_j$ by Theorem~\ref{thm:rules}.
\end{proof}



\subsection{Dialogues terminate} 
\begin{theorem}Given two initial deductions $D_1 = \deduction{F_1}{R_1}{\{T_1\}}$ and $D_2 = \deduction{F_2}{R_2}{\{T_2\}}$.  If there are only a finite number of terms that are provable in some proof tree for $F_i$ and $R_i$, then  any dialogue $D$ starting from $D_1$ and $D_2$ is finite.
\end{theorem}
\begin{proof}
We observe that the only statements in $D$ that do not terminate the dialogue are $initial(t)$, $\drule{l}{a}{c}$, $why(t)$ and $whynot(t)$.  $inital(t)$, $\drule{l}{a}{c}$ and $whynot(t)$ can only be made in response to the statement $why(t)$.  So it suffices to show that there are only a finite number of terms $t$ about which the statement $why(t)$ can appear in a dialogue and $why(t)$ can only be asked a finite number of times about each of these.  

Since we assume that there are only a finite number of terms provable in a proof tree for $F_1$ and $R_1$ and only a finite number of terms provable in some proof tree for $F_2$ and $R_2$ and we know that $why(t)$ is only asked if $t$ is provable in some proof three for either  $F_1$ and $R_1$  or  $F_2$ and $R_2$ then there are only a finite number of terms $t$ about which $why(t)$ is asked.

If the preceding statement is not $whynot(t)$ then $why(t)$ can only be asked if it has not been asked before, therefore we are only concerned with statements $why(t)$ that following statements $whynot(t)$.  However $whynot(t)$ may only be asked once in a dialogue and so $why(t)$ can only respond to this once, even if $why(t)$ has been asked before.  Therefore the dialogue terminates (either with a terminating statement or because no legal move can be made).
\end{proof}

Note this means that only some dialogues terminate and this depends on the structure of the facts and rules for each player.  However we know that many sets of facts and rules have this property.  

\subsection{When dialogues terminate, a difference is detected} 
Now we have to show that a dialogue terminates by someone saying $\mathit{different\_fact}(t, i, j)$ or $\mathit{different\_rule}(l, i, j)$ not just because the dialogue has run out of legal moves. 

\begin{theorem}
If the $k$th state in a legal sequence of dialogue states is $\langle S^1_{k}, S^2_{k}, s, i \rangle$ and $s \neq  \mathit{different\_fact}(t, i, j)$, $s \neq \mathit{different\_fact}(t, j, i)$ and $s \neq \mathit{different\_rule}(l, i, j)$ then there is a legal next dialogue state.
\end{theorem}
\begin{proof}

Recall that $S^i_k = \langle \deduction{F_i}{R_i}{{\cal T}_i}, Y_{ij}, F_{ij}, N_{ij}, YR_{ij} \rangle$

We establish that in all legal dialogue states one of the following must hold:
\begin{enumerate}
\item There exists $t \in N_{ij}$ that is provable in some proof tree, $T \in {\cal T}_i$ and $whynot(t)$ has not previously been asked.
\item There exists a $t \in Y_{ij}$ which is unprovable in some proof tree, $T \in {\cal T}_i$ and $why(t)$ has not previously been asked.
\item There exists a $t \in F_{ij}$ and $t \not\in F_i$
\item There exists some $l \in YR_{ij}$ such that there is no rule $\drule{l}{a}{c} \in R_i$.
\end{enumerate}
We observe that if at least one of the above holds then there is a legal next move.

We consider a proof by cases

\begin{enumerate}
\item {\bf Dialogue start.} In this case $Y_{12} = \{t\}$ and $N_{21} = \{t\}$ for some $t$ provable $T_2$ and unprovable in $T_1$ (or vice versa).  So either 1 or 2 above is true.

\item $s = initial(t)$.  This means $t \in F_j$ (for this to be legal utterance).  It can only be uttered in response to $why(t)$ said by $i$ so we also know that $t$ is provable in some $T \in {\cal T}_j$ ($j$ believes $t$) (corollary to theorem~\ref{thm:Ycorrect} ) and that $t$ is not provable in any proof tree for $F_i$ and $R_i$ (corollary 3 to theorem~\ref{thm:Ycorrect}).  Since $t$ is not provable  in any proof tree for $F_i$ and $R_i$  then $t \not\in F_i$ (otherwise $\pt{\{\node{t}{initial}\}}{\emptyset}$ would be a proof tree for $F_i$ and $R_i$) therefore after updating its state by adding $t$ to $F_{ij}$ there exists a $t \in F_{ij}$ and $t \not\in F_i$.

\item $s = \drule{l}{a}{t}$.  This can only be uttered in response to $why(t)$ so we know that  $t$ is provable in some $T \in {\cal T}_j$ and that $t$ is not provable in any proof tree for $F_i$ and $R_i$.  Since $t$ is not provable in any proof tree for $F_i$ and $R_i$ then there is no rule that would allow $i$ to deduce $t$ so either $\drule{l}{a}{t} \not\in R_i$ and after $l$ is added to $YR_{ij}$ our fifth property holds.  Or there is some term $t \in pos(a)$ where $t$ is not provable in any proof tree for $F_i$ and $R_i$ or some term  $t \in neg(a)$ which is provable in some proof tree for $F_i$ and $R_i$ (there is some difference in the antecedants).  

If there is some term $t \in pos(a)$ where $t$ is not provable in some proof tree for $F_i$ and $R_i$ then after adding $pos(a)$ to $Y_{ij}$ there exists a $t \in Y_{ij}$ which is provable in no proof tree for $F_j$ and $R_j$.
 
 If  there is a $t \in neg(a)$ which is provable in some proof tree for $F_i$ and $R_i$ then after adding $neg(a)$ to $N_{ij}$ there exists $t \in N_{ij}$ that is the root of a proof tree for $F_j$ and $R_j$ and is provable in that tree.
 
\item $s = why(t)$.  In this case $t \in Y_{ji}$, $t$ is unprovable in some $T \in {\cal T}_j$ and $t$ is added to $N_{ij}$. Since all terms in $Y_{ji}$ are provable in some $T \in {\cal T}_i$ (Theorem~\ref{thm:Ycorrect}) there exists $t \in N_{ij}$ that is provable in some proof tree for $F_j$ and $R_j$.

\item $s = whynot(t)$.  In this case $t \in N_{ji}$, $t$ is provable in some $T \in {\cal T}_j$, and $t$ is added to $Y_{ij}$. Since every term in $N_{ji}$ is unprovable in some $T \in {\cal T}_j$ (Theorem~\ref{thm:Ncorrect}), there exists a $t \in Y_{ij}$ which is provable in no proof tree for $F_j$ and $R_j$ (by Theorem~\ref{thm:unprovable}).
\end{enumerate}
\end{proof}

\begin{sloppypar}
\begin{theorem}
If a dialogue terminates the last statement is: $\mathit{different\_fact}(t, i, j)$ or $\mathit{different\_fact}(t, j, i)$ or $\mathit{different\_rule(l, i, j)}$
\end{theorem}
\end{sloppypar}
\begin{proof}This follows because we have shown that there is always a legal next move except in these cases.
\end{proof}

\bibliography{bluebook}
\bibliographystyle{apalike}
\end{document}
